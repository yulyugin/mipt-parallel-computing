% This file allows to produce either a separate PDF/PNG image
% See standalone documentation to understand underlying magic

\documentclass[tikz,convert={density=150,size=600,outext=.png}]{standalone}
\usetikzlibrary{shapes, calc, arrows, fit, positioning, decorations, patterns, decorations.pathreplacing, chains, snakes}
\input{../setup-web-fonts}
\input{../setup-packages}
\graphicspath{{../pictures/}} % path to pictures, trailing slash is mandatory.

% The actual drawing follows
\begin{document}
\begin{tikzpicture}[>=latex, font=\small]

% Level 0
\node [draw] (architecture) {\shortstack{Параллельные\\компьютерные архитектуры}};

% Level 1
\node [draw, below left= 0.5cm and 1cm of architecture, minimum width=2cm,
    minimum height=0.5cm] (sisd) {SISD};
\node [draw, below left= 0.5cm and -1.5cm of architecture, minimum width=2cm,
    minimum height=0.5cm] (simd) {SIMD};
\node [draw, below right= 0.5cm and -0.5cm of architecture, minimum width=2cm,
    minimum height=0.5cm] (misd) {MISD};
\node [draw, below right= 0.5cm and 2cm of architecture, minimum width=2cm,
    minimum height=0.5cm] (mimd) {MIMD};

% Level 1.5
\node [below=0cm of sisd] {\shortstack{\scriptsize{Архитектура}\\\scriptsize{фон Неймана}}};
\node [below=0cm of misd] {\scriptsize{?}};

% Level 2
\node [draw, below left=1cm and -0.5cm of simd] {\shortstack{Матричные\\процессоры}};
\node [draw, below right=1cm and -0.5cm of simd] {\shortstack{Векторные\\процессоры}};

\node [draw, below left=1cm and 0cm of mimd] (multiprocessor) {Мультипроцессоры};
\node [draw, below right=1cm and 0cm of mimd] (multicomputers) {Мультикомпьютеры};

\end{tikzpicture}

\end{document}
