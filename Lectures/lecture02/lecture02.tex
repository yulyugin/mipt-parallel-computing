% Compile with XeLaTeX, TeXLive 2013 or more recent
\documentclass{beamer}

% Base packages
\usepackage{fontspec}
\defaultfontfeatures{Scale=MatchLowercase, Mapping=tex-text}

\usepackage{amsfonts}
\usepackage{amsmath}
\usepackage{longtable}
\usepackage{csquotes}
\usepackage{standalone}

\usepackage{graphicx}
\graphicspath{{./images/}}

\usepackage{tikz}
\usetikzlibrary{shapes, calc, arrows, decorations, patterns, chains, fit, petri}
\usetikzlibrary{backgrounds, positioning, decorations.pathreplacing}

\usepackage{pgfplots}

\newcommand{\inputpicture}[1]{\input{../pictures/#1}}

\usepackage{tabularx}
\usepackage{multirow}

\usepackage{listings}
\lstset{language=C, basicstyle=\ttfamily, breaklines=true, keepspaces=true, keywordstyle=\color{blue}}

% Setup fonts
\newfontfamily\russianfont{CMU Serif}
\setromanfont{CMU Serif}
\setsansfont{CMU Sans Serif}
%\setmonofont{CMU Typewriter Text}

% Be able to insert hyperlinks
\usepackage{hyperref}
\hypersetup{colorlinks=true, linkcolor=black, filecolor=black, citecolor=black, urlcolor=blue , pdfauthor=Evgeny Yulyugin <yulyugin@gmail.com>, pdftitle=Параллельное программирование}
% \usepackage{url}

% Misc optional packages
\usepackage{underscore}
\usepackage{amsthm}

% A new command to mark not done places
\newcommand{\todo}[1][]{{\color{red}TODO\ #1}}

\newcommand{\abbr}{\textit{англ.}\ }

\newif\ifmipt
\newif\ifsbertech
\input{target}

\ifmipt
\subtitle{Курс <<Параллельное программирование>>}
\fi
\ifsbertech
\subtitle{Курс <<Инфраструктура многопроцессорных систем>>}
\fi
\subject{Лекция}
\author[Евгений Юлюгин]{Евгений Юлюгин \texorpdfstring{\\}{Lg} \small{\href{mailto:yulyugin@gmail.com}{yulyugin@gmail.com}}}
\date{\today}
\pgfdeclareimage[height=0.5cm]{mipt-logo}{common/mipt.png}
\logo{\pgfuseimage{mipt-logo}}

\typeout{Copyright 2014 Evgeny Yulyugin}

\usetheme{Berlin}
\setbeamertemplate{navigation symbols}{}%remove navigation symbols


\title{Title}

\begin{document}

\begin{frame}
\titlepage
\end{frame}

\section{Обзор}

\begin{frame}
\tableofcontents
\end{frame} 

\begin{frame}{На прошлой лекции}

\begin{itemize}
    \item Основы MPI.
\end{itemize}

\end{frame}

\section{Классификация архитектур вычислительных систем}

\begin{frame}{Классификация Флинна}

\begin{table}[htp]
    \begin{center}
    \begin{tabular}{|l|c|c|}
        \hline
                                & Single data   & Multiple data \\
        \hline
        Single instruction      & SISD          & SIMD \\
        \hline
        Multiple instruction    & MISD          & MIMD \\
        \hline
    \end{tabular}
    \end{center}
\end{table}

\end{frame}

\section{Состояние гонки}

\begin{frame}{Определение}

Состояние гонки (\abbr Race condition) --- ошибка в многопоточной программе, при которой работа приложения зависит от того, в каком порядке выполняются части кода.

Свое название получила от похожей ошибки проектирования электронных схем (Гонки сигналов).

Состояние гонки --- ошибка проявляющаяся в случайный момент времени.

\end{frame}

\begin{frame}[fragile]{Пример}

\begin{lstlisting}[language=C++,basicstyle=\ttfamily,keywordstyle=\color{blue},basicstyle=\footnotesize]
int N = 10;
int x = 0;
\end{lstlisting}

\begin{columns}[t]
    \begin{column}[T]{0.45\textwidth}
    \begin{lstlisting}[language=C++,basicstyle=\ttfamily,keywordstyle=\color{blue},basicstyle=\footnotesize]
// thread 0
for (i = 0; i < N; ++i) {
    x *= 2;
}
    \end{lstlisting}
    \end{column}
    \begin{column}[T]{0.45\textwidth}
    \begin{lstlisting}[language=C++,basicstyle=\ttfamily,keywordstyle=\color{blue},basicstyle=\footnotesize]
// thread 1
for (i = 0; i < N; ++i) {
    x += 2;
}
    \end{lstlisting}
    \end{column}
\end{columns}

\begin{lstlisting}[language=C++,basicstyle=\ttfamily,keywordstyle=\color{blue},basicstyle=\footnotesize]
printf("%d\n", x);
\end{lstlisting}

\end{frame}

\section{Синхронизация}

\begin{frame}{Deadlock}
\todo
\end{frame}

\begin{frame}{Livelock}
\todo
\end{frame}

\begin{frame}[fragile]{Алгоритм Деккера}

\begin{lstlisting}[language=C++,basicstyle=\ttfamily,keywordstyle=\color{blue},basicstyle=\scriptsize]
bool flag[2] = {false, false};
bool turn = false; // or true
\end{lstlisting}

\begin{columns}[t]
    \begin{column}[T]{0.45\textwidth}
        \begin{lstlisting}[language=C++,basicstyle=\ttfamily,keywordstyle=\color{blue},basicstyle=\scriptsize]
// thread 0
flag[0] = true;
while (flag[1]) {
    if (turn) {
        flag[0] = false;
        while (turn);
        flag[0] = true;
    }
}

// critical section
//...
turn = true;
flag[0] = false;
// end of critical section
// ...
        \end{lstlisting}
    \end{column}
    \begin{column}[T]{0.45\textwidth}
        \begin{lstlisting}[language=C++,basicstyle=\ttfamily,keywordstyle=\color{blue},basicstyle=\scriptsize]
// thread 1
flag[1] = true;
while (flag[0]) {
    if (!turn) {
        flag[1] = false;
        while (!turn);
        flag[1] = true;
    }
}

// critical section
//...
turn = false;
flag[0] = false;
// end of critical section
// ...
        \end{lstlisting}
    \end{column}
\end{columns}

\end{frame}

\section{Конец}
% The final "thank you" frame 

\begin{frame}{Задания}
\end{frame}

\begin{frame}{На следующей лекции}
\end{frame}

\begin{frame}

{\huge{Спасибо за внимание!}\par}

\vfill

\tiny{\textit{Замечание}: все торговые марки и логотипы, использованные в данном материале, являются собственностью их владельцев. Представленная здесь точка зрения отражает личное мнение автора, не выступающего от лица какой-либо организации.}

\end{frame}

\end{document}