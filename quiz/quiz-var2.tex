% Compile with XeLaTeX only
%
%\documentclass[a4paper, 12pt, addpoints, answers]{exam}
\documentclass[a4paper, 12pt, addpoints]{exam}

\pointpoints{б.}{б.}
\bonuspointpoints{б.}{б.}

\def \variant {2}

\usepackage{fontspec}

\usepackage{xunicode} % some extra unicode support
\usepackage{xltxtra}

\usepackage{amsfonts}
\usepackage{amsmath}

\usepackage{csquotes}

\usepackage{polyglossia}
\defaultfontfeatures{Scale=MatchLowercase, Mapping=tex-text}

\newfontfamily\russianfont{CMU Serif}
\setromanfont{CMU Serif}
\setsansfont{CMU Sans Serif}
\setmonofont{CMU Typewriter Text}

\setdefaultlanguage[spelling=modern]{russian}
\setotherlanguage{english}
\newcommand{\todo}{{\color{red}TODO}\ }
\usepackage{graphicx}

\tolerance=9999 % let the text underfull be ugly as hell, nobody cares.
\emergencystretch=3cm

\usepackage{hyperref}
\hypersetup{colorlinks=true, linkcolor=black, filecolor=black, citecolor=black, urlcolor=black , pdfauthor=Evgeny Yulyugin <yulyugin@gmail.com>, pdftitle=Контрольная по курсу «Введение в распараллеливание алгоритмов и программ»}

\usepackage{footnpag}
\usepackage{indentfirst}
\usepackage{underscore}
\usepackage{url}

\usepackage{listings}
\lstset{basicstyle=\footnotesize\ttfamily, breaklines=true, keepspaces=true}

\usepackage{nameref}
\usepackage{amsthm}
\usepackage{enumitem} % continue enumeration 
\usepackage{subfigure}
\usepackage{mdwlist} % compact itemize lists environment

\setcounter{tocdepth}{2}

\renewcommand{\solutiontitle}{\noindent\textbf{Ответ:}\enspace}

\title{Введение в распараллеливание алгоритмов и программ. Контрольная работа. Вариант \textnumero \variant}
\author{}
% \newcommand{\testdate}{\today}
\newcommand{\testdate}{11 декабря 2015 г.}
\date{\testdate}

\typeout{Copyright 2015 Evgeny Yulyugin and the contributors.}
\begin{document}

\pagestyle{headandfoot}
\runningheadrule
\firstpageheader{Вариант \textnumero \variant}{ }{\testdate}
\runningheader{Вариант \textnumero \variant}{ }{\testdate}

\coverfooter{}{Стр. \thepage\ из \numpages}{}
\firstpagefooter{}{Стр. \thepage\ из \numpages}{}
\runningfooter{}{Стр. \thepage\ из \numpages}{}

\maketitle\thispagestyle{headandfoot} % a hack to have proper footers and headers on the first page as well

\medskip
\makebox[0.9\textwidth]{Ф.И.О.\enspace\hrulefill}
\medskip

\makebox[0.9\textwidth]{Группа\enspace\hrulefill}
\medskip

% Grade tables
\hqword{Вопрос}
\hpword{Баллов}
\hsword{Результат}
\htword{Сумма}
\vqword{Вопрос}
\vpword{Баллов}
\vsword{Результат}
\vtword{Сумма}
\cellwidth{0.5em}
\gradetablestretch{1.8}
\begin{center}
\tiny
%\gradetable[h][questions]
\partialgradetable{multiplechoice}[h][questions]\\
\partialgradetable{fullquestions}[h][questions]\\
\end{center}

\medskip

% Introduce several grading range so that grading tables are short enough to fit on the page

\begingradingrange{multiplechoice} 

\begin{questions}
\question[1] Выберете верное утверждение. Для следующего цикла:
\begin{lstlisting}
for (i = 1; i < N; i++) {
  a[i] = f(a[i + 1]);
}
\end{lstlisting}
\begin{choices}
    \choice Нет зависимостей между итереациями цикла. Может быть выполнен на $\leq$ \texttt{N} исполнителях.
    \choice Истинная зависимость. Может быть выполнен на $\leq$ \texttt{N} исполнителях.
    \choice Истинная зависимость. Может быть выполнен на одном исполнителе.
    \correctchoice Антизависимость. Может быть выполнен на $\leq$ \texttt{N} исполнителях.
    \choice Антизависимость. Может быть выполнен на одном исполнителе.
\end{choices}

\question[1] Совмещение вычислений и операций передачи данных в программе:
\begin{choices}
    \choice возможно при использовании синхронных операций передачи данных
    \correctchoice возможно при использовании асинхронных операций передачи данных
    \choice невозможно
\end{choices}

\question[1] Какая зависимость по данным сложнее всего распараллеливается?
\begin{choices}
    \correctchoice Потоковая зависимость
    \choice Антизависимость
    \choice Зависимость по выходным данным
\end{choices}

\question[1] В кластерной вычислительной системе время передачи данных между процессорами определяется:
\begin{choices}
    \choice типом элементов передаваемых данных.
    \correctchoice латентностью среды передачи данных и объемом передаваемых данных.
    \choice расстоянием между вычислительными узлами.
    \choice Всеми вышеперечисленными факторами.
\end{choices}

\question[1] Может ли одна и та же программа на 4-х ядерном процессоре работать медленнее чем на одноядерном?
\begin{choices}
    \correctchoice да
    \choice нет
\end{choices}

\question[1] Что такое атомарная операция?
\begin{choices}
    \choice операция над атомами.
    \choice операция ввода-вывода.
    \choice базовая арифметическая операция.
    \correctchoice операция, которую невозможно прерывать.
\end{choices}

\question[1] Какая архитектура получила наибольшее распространение среди суперкомпьютеров?
\begin{choices}
    \choice MISD,
    \choice SIMD,
    \choice SISD,
    \correctchoice MIMD,
    \choice Все архитектуры одинакого часто используются.
\end{choices}

\question[1] Какие зависимости по данным допускают эффективное распараллеливание цикла?
\begin{choices}
    \choice растояние зависимости d неопределено
    \correctchoice растояние зависимости d=0
    \choice растояние зависимости d=1
\end{choices}

\question[1] Чему равен вектор направлений для вектора расстояний G=(0,-1)?
\begin{choices}
    \choice (=, =)
    \choice (=, <)
    \correctchoice (=, >)
\end{choices}

\question[1] В каких векторах расстояний есть истинная зависимость хотя бы по одной координате ?
\begin{choices}
    \correctchoice (1,-1)
    \correctchoice (1,1)
    \choice (-1,0)
\end{choices}

\question[1] Данные могут попадать в кэш при следующих операциях:
\begin{choices}
    \correctchoice чтение памяти (load),
    \correctchoice запись в память (store),
    \choice арифметические операции,
    \choice операции с числами с плавающей запятой,
    \correctchoice предвыборка данных (prefetch),
    \correctchoice загрузка инструкции (fetch),
    \choice инвалидация линии (invalidate).
\end{choices}

\question[1] Выберите правильные варианты окончания: линия данных с фиксированным адресом
\begin{choices}
    \choice всегда попадает в одну и ту же ячейку кэша,
    \correctchoice всегда попадает в один и тот же сет,
    \choice может быть сохранена в любой ячейке кэша.
\end{choices}

\question[1] Как организована память в массивно-параллельных вычислительных системах?
\begin{choices}
    \choice общая память,
    \choice NUMA память,
    \correctchoice распределенная память.
\end{choices}

\question[1] С какого предложения начинается параллельный блок в программе на языке C?
\begin{choices}
    \choice \begin{verbatim}#pragma omp for\end{verbatim}
    \choice \begin{verbatim}#parallel\end{verbatim}
    \choice \begin{verbatim}#omp parallel\end{verbatim}
    \choice \begin{verbatim}#include <omp.h>\end{verbatim}
    \correctchoice \begin{verbatim}#pragma omp parallel\end{verbatim}
    \choice \begin{verbatim}omp_parallel();\end{verbatim}
\end{choices}

\endgradingrange{multiplechoice}

\begingradingrange{fullquestions}

\question[3] Определите понятие «критическая секция».
\begin{solution}[2cm]
Участок кода программы, в котором производится доступ к общему ресурсу.
\end{solution}

\question[3] Определите понятие эффективности параллельной алгоритма.
\begin{solution}[2cm]
\todo
\end{solution}

\question[3] Сформулируйте закон Амдала.
\begin{solution}[2cm]
\todo
\end{solution}

\question[3] Сформулируйте условия Бернстайна.
\begin{solution}[2cm]
\todo
\end{solution}

\question[3] Определите понятие расстояния зависимости.
\begin{solution}[2cm]
\todo
\end{solution}

\question[3] Проанализируйте возможность распараллеливания следующего цикал:
\begin{lstlisting}
for (int i=0; i<N; ++i) { a[i] = a[i+4]*tan(a[i -1]);}
\end{lstlisting}
\begin{solution}[2cm]
\todo
\end{solution}

\question[3] Сформулируйте закон Мура для многопроцессорных систем.
\begin{solution}[2cm]
Удвоение количиства ядер на процессоре происходит каждые
полтора–два года.
\end{solution}

\question[3] Как скомпилировать и запустить программу, написанную с использованием MPI?
\begin{solution}[2cm]
\end{solution}
\endgradingrange{fullquestions} 

% Some extra blank pages
% \newpage
% \phantom{Blank page}
%\newpage
%\phantom{Blank page}

\end{questions}
\end{document}
