% Compile with XeLaTeX only
%
%\documentclass[a4paper, 12pt, addpoints, answers]{exam}
\documentclass[a4paper, 12pt, addpoints]{exam}

\pointpoints{б.}{б.}
\bonuspointpoints{б.}{б.}

\def \variant {1}

\usepackage{fontspec}

\usepackage{xunicode} % some extra unicode support
\usepackage{xltxtra}

\usepackage{amsfonts}
\usepackage{amsmath}

\usepackage{csquotes}

\usepackage{polyglossia}
\defaultfontfeatures{Scale=MatchLowercase, Mapping=tex-text}

\newfontfamily\russianfont{CMU Serif}
\setromanfont{CMU Serif}
\setsansfont{CMU Sans Serif}
\setmonofont{CMU Typewriter Text}

\setdefaultlanguage[spelling=modern]{russian}
\setotherlanguage{english}
\newcommand{\todo}{{\color{red}TODO}\ }
\usepackage{graphicx}

\tolerance=9999 % let the text underfull be ugly as hell, nobody cares.
\emergencystretch=3cm

\usepackage{hyperref}
\hypersetup{colorlinks=true, linkcolor=black, filecolor=black, citecolor=black, urlcolor=black , pdfauthor=Evgeny Yulyugin <yulyugin@gmail.com>, pdftitle=Контрольная по курсу «Введение в распараллеливание алгоритмов и программ»}

\usepackage{footnpag}
\usepackage{indentfirst}
\usepackage{underscore}
\usepackage{url}

\usepackage{listings}
\lstset{basicstyle=\footnotesize\ttfamily, breaklines=true, keepspaces=true}

\usepackage{nameref}
\usepackage{amsthm}
\usepackage{enumitem} % continue enumeration 
\usepackage{subfigure}
\usepackage{mdwlist} % compact itemize lists environment

\setcounter{tocdepth}{2}

\renewcommand{\solutiontitle}{\noindent\textbf{Ответ:}\enspace}

\title{Введение в распараллеливание алгоритмов и программ. Контрольная работа. Вариант \textnumero \variant}
\author{}
% \newcommand{\testdate}{\today}
\newcommand{\testdate}{11 декабря 2015 г.}
\date{\testdate}

\typeout{Copyright 2015 Evgeny Yulyugin and the contributors.}
\begin{document}

\pagestyle{headandfoot}
\runningheadrule
\firstpageheader{Вариант \textnumero \variant}{ }{\testdate}
\runningheader{Вариант \textnumero \variant}{ }{\testdate}

\coverfooter{}{Стр. \thepage\ из \numpages}{}
\firstpagefooter{}{Стр. \thepage\ из \numpages}{}
\runningfooter{}{Стр. \thepage\ из \numpages}{}

\maketitle\thispagestyle{headandfoot} % a hack to have proper footers and headers on the first page as well

\medskip
\makebox[0.9\textwidth]{Ф.И.О.\enspace\hrulefill}
\medskip

\makebox[0.9\textwidth]{Группа\enspace\hrulefill}
\medskip

% Grade tables
\hqword{Вопрос}
\hpword{Баллов}
\hsword{Результат}
\htword{Сумма}
\vqword{Вопрос}
\vpword{Баллов}
\vsword{Результат}
\vtword{Сумма}
\cellwidth{0.5em}
\gradetablestretch{1.8}
\begin{center}
\tiny
%\gradetable[h][questions]
\partialgradetable{multiplechoice}[h][questions]\\
\partialgradetable{fullquestions}[h][questions]\\
\end{center}

\medskip

% Introduce several grading range so that grading tables are short enough to fit on the page

\begingradingrange{multiplechoice} 

\begin{questions}
\question[1]
Выберете верное утверждение. Для следующего цикла:
\begin{lstlisting}
for (i = 1; i < N; i++) {
  a[i] = f(a[i - 1]);
}
\end{lstlisting}
\begin{choices}
    \choice Нет зависимостей между итереациями цикла. Может быть выполнен на $\leq$ \texttt{N} исполнителях.
    \choice Истинная зависимость. Может быть выполнен на $\leq$ \texttt{N} исполнителях.
    \correctchoice Истинная зависимость. Может быть выполнен на одном исполнителе.
    \choice Антизависимость. Может быть выполнен на $\leq$ \texttt{N} исполнителях.
    \choice Антизависимость. Может быть выполнен на одном исполнителе.
\end{choices}

\question[1] Какое максимальное количество исполнителей может быть эффективно задействовано для выполнения данного цикла
\begin{lstlisting}
for (i = 2; i < N; i++) {
  a[i] = f(a[i - 2]);
}
\end{lstlisting}
\begin{choices}
    \choice 1.
    \choice N.
    \correctchoice 2.
    \choice N - 2.
\end{choices}

\question[1] Для чего использутеся понятие эффективности параллельного алгоритма?
\begin{choices}
    \choice Для оценки скорости работы параллельного алгоритма.
    \correctchoice Для оценки масштабируемости параллельного алгоритма.
    \choice Для определения области применимости параллельного алгоритма.
\end{choices}

\question[1] В кластерной вычислительной системе время передачи данных между процессорами определяется:
\begin{choices}
    \choice типом элементов передаваемых данных.
    \correctchoice латентностью среды передачи данных и объемом передаваемых данных.
    \choice расстоянием между вычислительными узлами.
    \choice всеми вышеперечисленными факторами.
\end{choices}

\question[1] Чем отличаются многоядерная и многопроцессорная архитектуры?
\begin{choices}
    \choice в многопроцессорной архитектуре общий кеш,
    \choice в многопроцессорной архитектуре общая оперативная память,
    \choice в многоядерной архитектуре общая оперативная память,
    \correctchoice в многоядерной архитектуре общий кеш.
\end{choices}

\question[1] С какой зависимостью по данным в цикле можно справиться, скопировав данные на каждый процессор?
\begin{choices}
    \correctchoice антизависимость,
    \choice зависимость по выходным данным,
    \choice потоковая зависимость.
\end{choices}

\question[1] Чему равен вектор направлений для вектора расстояний G=(0,1) ?
\begin{choices}
    \choice (=, =)
    \correctchoice (=, <)
    \choice (=, >)
\end{choices}

\question[1] В каких векторах расстояний есть антизависимость хотя бы по одной координате ?
\begin{choices}
    \correctchoice (1,-1)
    \choice (1,1)
    \correctchoice (-1,0)
\end{choices}

\question[1] Выберите правильные варианты продолжения фразы: использование кэшей
при работе приложения целесообразно, если
\begin{choices}
    \correctchoice программа показывает временную локальность доступов,
    \choice программа не обращается в оперативную память,
    \choice программа работает с очень большим объёмом данных,
    \correctchoice программа показывает пространственную локальность доступов,
    \choice программа работает с объёмом данных, меньшим ёмкости кэша
\end{choices}

\question[1] Выберите правильные варианты.
\begin{choices}
    \choice Темпы роста скорости оперативной памяти и процессоров одинаковы с 80-х годов ХХ века.
    \choice Темп роста скорости оперативной памяти опережает темпы роста скорости работы процессоров.
    \correctchoice Темп роста скорости процессоров опережает темпы роста скорости оперативной памяти. 
\end{choices}

\question[1] Наличие каких зависимостей в цикле допускает эффективное распараллеливание?
\begin{choices}
    \choice истинная зависимость,
    \correctchoice антизависимость,
    \correctchoice истинная зависимость с большим по модулю расстоянием зависимости.
\end{choices}

\question[1] Какая зависимость по данным сложнее всего распараллеливается?
\begin{choices}
    \correctchoice Потоковая зависимость
    \choice Антизависимость
    \choice Зависимость по выходным данным
\end{choices}

\question[1] Сколько вычислительных потоков будет задействовано при выполнении данного куска кода?
\begin{lstlisting}
omp_set_num_threads(2);
#pragma omp for
for (i = 0; i < N; i++)
    printf("%d: %d\n", id, i);
\end{lstlisting}
\begin{choices}
    \choice 2,
    \choice OMP_NUM_THREADS,
    \correctchoice 1,
    \choice N.
\end{choices}

\question[1] Зачем нужна синхронизация типа critical?
\begin{choices}
    \choice для определения изменения порядка выполнения циклов,
    \choice для правильного определения выполнения структурного блока в параллельном режиме,
    \choice для ускорения процесса выполнения задачи,
    \correctchoice для задания структурных блоков, выполняющихся только в одном потоке из всего набора параллельных потоков.
\end{choices}

\endgradingrange{multiplechoice}

\begingradingrange{fullquestions}

\question[3] Определите понятие ускорения параллельного алгоритма.
\begin{solution}[2cm]
Ускорением  параллельного  алгоритма  называют  отношение времени  выполнения 
лучшего последовательного алгоритмам к времени выполнения параллельного
алгоритма.
\end{solution}

\question[3] Определите понятие «атомарная операция».
\begin{solution}[2cm]
Операция, выполняющаяся как единое целое, либо не выполняющиеся вовсе.
\end{solution}

\question[3] Может ли эффективность параллельной программы превышать единицу? Ответ обоснуйте.
\begin{solution}[2cm]
\todo
\end{solution}

\question[3] Сформулируйте закон Амдала.
\begin{solution}[2cm]
\todo
\end{solution}

\question[3] Определите понятие расстояния зависимости.
\begin{solution}[2cm]
\todo
\end{solution}

\question[3] Проанализируйте возможность распараллеливания следующего цикал:
\begin{lstlisting}
for (int i=0; i<N; ++i) { a[i] = d[i] + 5*i; c[i] = a[2*i] * 2;}
\end{lstlisting}
\begin{solution}[2cm]
\todo
\end{solution}

\question[3] Сформулируйте закон Мура.
\begin{solution}[2cm]
Количество полупроводниковых элементов на кристалле и
производительность процессора удваивается в среднем каждые
полтора–два года.
\end{solution}

\question[3] Как скомпилировать и запустить программу, написанную с использованием OpenMP?
\begin{solution}[2cm]
\end{solution}

\endgradingrange{fullquestions}

% Some extra blank pages
% \newpage
% \phantom{Blank page}
%\newpage
%\phantom{Blank page}

\end{questions}
\end{document}
