% Compile with XeLaTeX, TeXLive 2013 or more recent
\documentclass{beamer}

% Base packages
\usepackage{fontspec}
\defaultfontfeatures{Scale=MatchLowercase, Mapping=tex-text}

\usepackage{amsfonts}
\usepackage{amsmath}
\usepackage{longtable}
\usepackage{csquotes}
\usepackage{standalone}

\usepackage{graphicx}
\graphicspath{{./images/}}

\usepackage{tikz}
\usetikzlibrary{shapes, calc, arrows, decorations, patterns, chains, fit, petri}
\usetikzlibrary{backgrounds, positioning, decorations.pathreplacing}

\newcommand{\inputpicture}[1]{\input{../pictures/#1}}

\usepackage{tabularx}
\usepackage{multirow}

\usepackage{listings}
\lstset{language=C, basicstyle=\ttfamily, breaklines=true, keepspaces=true, keywordstyle=\color{blue}}

% Setup fonts
\newfontfamily\russianfont{CMU Serif}
\setromanfont{CMU Serif}
\setsansfont{CMU Sans Serif}
%\setmonofont{CMU Typewriter Text}

% Be able to insert hyperlinks
\usepackage{hyperref}
\hypersetup{colorlinks=true, linkcolor=black, filecolor=black, citecolor=black, urlcolor=blue , pdfauthor=Evgeny Yulyugin <yulyugin@gmail.com>, pdftitle=Параллельное программирование}
% \usepackage{url}

% Misc optional packages
\usepackage{underscore}
\usepackage{amsthm}

% A new command to mark not done places
\newcommand{\todo}[1][]{{\color{red}TODO\ #1}}

\newcommand{\abbr}{\textit{англ.}\ }

\newif\ifmipt
\newif\ifsbertech
\input{target}

\ifmipt
\subtitle{Курс <<Параллельное программирование>>}
\fi
\ifsbertech
\subtitle{Курс <<Инфраструктура многопроцессорных систем>>}
\fi
\subject{Лекция}
\author[Евгений Юлюгин]{Евгений Юлюгин \texorpdfstring{\\}{Lg} \small{\href{mailto:yulyugin@gmail.com}{yulyugin@gmail.com}}}
\date{\today}
\pgfdeclareimage[height=0.5cm]{mipt-logo}{common/mipt.png}
\logo{\pgfuseimage{mipt-logo}}

\typeout{Copyright 2014 Evgeny Yulyugin}

\usetheme{Berlin}
\setbeamertemplate{navigation symbols}{}%remove navigation symbols


\title{Зависимости в циклах и их анализ на параллельность.}

\begin{document}

\begin{frame}
\titlepage
\end{frame}

\section*{Обзор}

\begin{frame}{На прошлой лекции}
\begin{itemize}
    \item Ускорение и эффективность параллельного алгоритма.
    \item Закон Амдала.
    \item Область применения закона Амдала.
\end{itemize}
\end{frame}

\begin{frame}{На этой лекции}
\tableofcontents
\end{frame}

\section{Виды зависимостей}

\begin{frame}{Виды зависимостей}
\begin{enumerate}
    \item Зависимости по данным,
    \pause
    \item Зависимости по управлению,
    \pause
    \item Зависимости по ресурсам.
\end{enumerate}
\end{frame}

\section{Зависимости по данным}

\begin{frame}{Зависимости по данным}
Виды зависимостей по данным:

\begin{itemize}
    \item Зависимость по выходным данным,
    \item Зависимость по входным данным,
    \item Антизависимость,
    \item Истинная или потоковая зависимость.
\end{itemize}
\end{frame}

\begin{frame}{Зависимости по данным}
\begin{itemize}
  \item WAW --- Зависимость по выходным данным,
  \item RAR --- Зависимость по входным данным,
  \item WAR --- Антизависимость,
  \item RAW --- Истинная или потоковая зависимость.
\end{itemize}
\end{frame}

\begin{frame}[fragile]{Зависимость по выходным данным}
Output dependence.

\pause\bigskip

\begin{lstlisting}
x = 2 * y + 5;
x = 3 + k;
\end{lstlisting}

Мешает ли распараллеливанию?

\pause\bigskip

\begin{lstlisting}
x1 = 2 * y + 5;
x2 = 3 + k;
\end{lstlisting}
\end{frame}

\begin{frame}[fragile]{Зависимость по входным данным}
Input dependence.

\pause\bigskip

\begin{lstlisting}
y = x + 4;
z = x + 5;
\end{lstlisting}
\end{frame}

\begin{frame}[fragile]{Антизависимость}
Antidependence.

\pause\bigskip

\begin{lstlisting}
x = 2 * y + 1;
y = z + 2;
\end{lstlisting}

Мешает ли распараллеливанию?

\pause

\begin{lstlisting}
y1 = y;
x = 2 * y1 + 1;
y = z + 2;
\end{lstlisting}
\end{frame}

\begin{frame}[fragile]{Истинная или потоковая зависимость}
Flow (True) dependence.

\pause\bigskip

\begin{lstlisting}
x = 2 + z;
y = 4 + x;
\end{lstlisting}

\pause\bigskip

Мешает ли распараллеливанию?

\pause\bigskip

Да.
\end{frame}

\section{Индексный анализ зависимостей}

\begin{frame}[fragile]{Зависимости в невложенных циклах}
\begin{lstlisting}
for (i = 1; i < N; i++) {
  A[f(i)] = ...;
  ... = A[g(i)];
}
\end{lstlisting}

\pause

Раскручиваем цикл:

\bigskip

$S_1^1: A[f(1)] = ...;$

$S_2^1: ... = A[g(1)];$

<...>

$S_1^{N - 1}: A[f(N - 1)] = ...;$

$S_2^{N - 1}: ... = A[g(N - 1)];$

\pause\bigskip

Если $\exists k, l: 0 < k, l < N$ и $f(k) = g(l)$, то есть зависимость.
\end{frame}

\begin{frame}{Расстояние зависимости}
$S_1^k$ --- источник (\abbr source) зависимости, $S_2^l$ --- сток (\abbr sink)
зависимости.

\pause

Для того, чтоб узнать, можно ли распараллелить цикл необходимо вычислить:

$d = l - k$ --- расстояние зависимости.

\pause

\begin{itemize}
    \item d < 0 $\Rightarrow$ антизависимость.
    \item d = 0 $\Rightarrow$ зависимостей между итерациями нет (\abbr loop
    independent dependence).
    \item d > 0 $\Rightarrow$ истиная зависимость между операциями цикла,
    исполняющимися на разных итерациях,
\end{itemize}
\end{frame}

\begin{frame}[fragile]{Пример}
\begin{lstlisting}
for (i = 0; i < N; ++i) {
  a[i] = f(a[i + 1]);
}
\end{lstlisting}

Разворачиваем цикл:

\begin{enumerate}
  \item \texttt{a[1] = f(a[2]);}
  \item \texttt{a[2] = f(a[3]);}
  \item \texttt{a[3] = f(a[4]);}
\end{enumerate}

d = -1 $\Rightarrow$ в цикле есть антизависимость и от нее можно избавиться
копированием данных. Можно выполнить на $\leq$ \texttt{N} потоках.
\end{frame}

\begin{frame}[fragile]{Пример}

\begin{lstlisting}
for (i = 0; i < N; i++) {
  a[i] = f(a[i]);
}
\end{lstlisting}

\pause\bigskip

d = 0 $\Rightarrow$ зависимости нет. Можно выполнить на $\leq$ \texttt{N} потоках.
\end{frame}

\begin{frame}[fragile]{Пример}
\begin{lstlisting}
for (i = 1; i < N; i++) {
  a[i] = f(a[i - 1]);
}
\end{lstlisting}

\pause\bigskip

d = 1 $\Rightarrow$ истинная зависимость. Можно выполнить только в один поток.
\end{frame}

\begin{frame}[fragile]{Пример}
\begin{lstlisting}
for (i = 2; i < N; i++) {
  a[i] = f(a[i - 2]);
}
\end{lstlisting}

\pause\bigskip

d = 2 $\Rightarrow$ истинная зависимость. Можно на двух потоках.
\end{frame}

\begin{frame}{Зависимости во вложенных циклах}
\end{frame}

\section*{Конец}

\begin{frame}{На следующей лекции}
\begin{itemize}
    \item Цикл работы центрального процессора,
    \item Идеальный конвейер,
    \item Реальный конвейер.
\end{itemize}
\end{frame}

\begin{frame}

{\huge{Спасибо за внимание!}\par}

\vfill

\tiny{\textit{Замечание}: все торговые марки и логотипы, использованные в данном материале, являются собственностью их владельцев. Представленная здесь точка зрения отражает личное мнение автора, не выступающего от лица какой-либо организации.}

\end{frame}

\end{document}
