% Compile with XeLaTeX, TeXLive 2013 or more recent
\documentclass{beamer}

% Base packages
\usepackage{fontspec}
\defaultfontfeatures{Scale=MatchLowercase, Mapping=tex-text}

\usepackage{amsfonts}
\usepackage{amsmath}
\usepackage{longtable}
\usepackage{csquotes}
\usepackage{standalone}

\usepackage{graphicx}
\graphicspath{{./images/}}

\usepackage{tikz}
\usetikzlibrary{shapes, calc, arrows, decorations, patterns, chains, fit, petri}
\usetikzlibrary{backgrounds, positioning, decorations.pathreplacing}

\newcommand{\inputpicture}[1]{\input{../pictures/#1}}

\usepackage{tabularx}
\usepackage{multirow}

\usepackage{listings}
\lstset{language=C, basicstyle=\ttfamily, breaklines=true, keepspaces=true, keywordstyle=\color{blue}}

% Setup fonts
\newfontfamily\russianfont{CMU Serif}
\setromanfont{CMU Serif}
\setsansfont{CMU Sans Serif}
%\setmonofont{CMU Typewriter Text}

% Be able to insert hyperlinks
\usepackage{hyperref}
\hypersetup{colorlinks=true, linkcolor=black, filecolor=black, citecolor=black, urlcolor=blue , pdfauthor=Evgeny Yulyugin <yulyugin@gmail.com>, pdftitle=Параллельное программирование}
% \usepackage{url}

% Misc optional packages
\usepackage{underscore}
\usepackage{amsthm}

% A new command to mark not done places
\newcommand{\todo}[1][]{{\color{red}TODO\ #1}}

\newcommand{\abbr}{\textit{англ.}\ }

\newif\ifmipt
\newif\ifsbertech
\input{target}

\ifmipt
\subtitle{Курс <<Параллельное программирование>>}
\fi
\ifsbertech
\subtitle{Курс <<Инфраструктура многопроцессорных систем>>}
\fi
\subject{Лекция}
\author[Евгений Юлюгин]{Евгений Юлюгин \texorpdfstring{\\}{Lg} \small{\href{mailto:yulyugin@gmail.com}{yulyugin@gmail.com}}}
\date{\today}
\pgfdeclareimage[height=0.5cm]{mipt-logo}{common/mipt.png}
\logo{\pgfuseimage{mipt-logo}}

\typeout{Copyright 2014 Evgeny Yulyugin}

\usetheme{Berlin}
\setbeamertemplate{navigation symbols}{}%remove navigation symbols


\title{Вычислительный конвейер}

\begin{document}

\begin{frame}
\titlepage
\end{frame}

\section*{Обзор}

\begin{frame}{На прошлой лекции}
\end{frame}

\begin{frame}{На этой лекции}
\tableofcontents
\end{frame}

\section{Цикл работы процессора}

\begin{frame}{Цикл работы процессора}
\centering
\inputpicture{cpu-cycle}
\end{frame}

\begin{frame}{Стадии конвейра}
\begin{itemize}
    \item Получение инструкции (Instruction Fetch, IF),
    \item Декодирование (Instruction decode, ID),
    \item Считывание операндов (Read operands, RO),
    \item Выполнение (Execute, EX),
    \item Запись результатов (Write back, WB).
\end{itemize}
\end{frame}

\begin{frame}
\centering
\resizebox{10cm}{2cm}{
\inputpicture{non-pipeline}
}
\end{frame}

\section{Идеальный конвейер}

\begin{frame}{Конвейер}
\centering
\inputpicture{pipeline}
\end{frame}

\begin{frame}{Оценка производительности}
Время выполнения при последовательной обработке:
\begin{displaymath}
T_{sequential} = (T_{IF} + T_{ID} + T_{RO} + T_{EX} + T_{WB}) * N
\end{displaymath}
\vfill\pause
Время выполнения при конвейерном исполнении:
\begin{displaymath}
T_{pipeline} = 5 * T + (N - 1) * T
\end{displaymath}
\begin{displaymath}
T = max \{T_{IF}, T_{ID}, T_{RO}, T_{EX}, T_{WB}\} + \Delta t
\end{displaymath}
\begin{center}
$\Delta t$ --- время, необходимое для передачи команд\\с одной стадии на другую.
\end{center}
\end{frame}

\begin{frame}{Пример}
Времена выполнения стадий:

$T_{IF} = 4$, $T_{ID} = 3$, $T_{RO} = 4$, $T_{EX} = 4$, $T_{WB} = 3$
\vfill
Накладные расходы:

$\Delta t = 1$
\vfill
Время выполнения стадии конвейера:

$T = \pause5$
\end{frame}

\begin{frame}{Пример}
\begin{table}[htpb]
    \centering
    \begin{tabular}{|l|r|r|}
    \hline
    \multirow{2}{*}{Количество команд}   &   \multicolumn{2}{c|}{Время обработки} \\
    \cline{2-3}
                        &   последовательной    &   конвейерной \\
    \hline
    1                   &   18                  &   25          \\
    \hline
    2                   &   36                  &   30          \\
    \hline
    10                  &   180                 &   70          \\
    \hline
    100                 &   1800                &   520         \\
    \hline
    \end{tabular}
\end{table}
\end{frame}

\section*{Конец}
% The final "thank you" frame

\begin{frame}{Задания}
\end{frame}

\begin{frame}{На следующей лекции}
\end{frame}

\begin{frame}

{\huge{Спасибо за внимание!}\par}

\vfill

\tiny{\textit{Замечание}: все торговые марки и логотипы, использованные в данном материале, являются собственностью их владельцев. Представленная здесь точка зрения отражает личное мнение автора, не выступающего от лица какой-либо организации.}

\end{frame}

\end{document}
