% Compile with XeLaTeX, TeXLive 2013 or more recent
\documentclass{beamer}

% Base packages
\usepackage{fontspec}


\usepackage{amsfonts}
\usepackage{amsmath}
\usepackage{longtable}
\usepackage{csquotes}
\usepackage{standalone}

\usepackage{graphicx}
\graphicspath{{./images/}}

\usepackage{tikz}
\usetikzlibrary{arrows,decorations.pathmorphing,backgrounds,positioning,fit,petri}

\usepackage{listings}
\lstset{language=C, basicstyle=\ttfamily, breaklines=true, keepspaces=true, keywordstyle=\color{blue}}

% Setup Russian hyphenation
\usepackage{polyglossia}
\setdefaultlanguage[spelling=modern]{russian} % for polyglossia
\setotherlanguage{english} % for polyglossia

% Setup fonts
\newfontfamily\russianfont{CMU Serif}
\setromanfont{CMU Serif}
\setsansfont{CMU Sans Serif}
\setmonofont{CMU Typewriter Text}

% Be able to insert hyperlinks
\usepackage{hyperref}
\hypersetup{colorlinks=true, linkcolor=black, filecolor=black, citecolor=black, urlcolor=blue , pdfauthor=Evgeny Yulyugin <yulyugin@gmail.com>, pdftitle=Параллельное программирование}
% \usepackage{url}

% Misc optional packages
\usepackage{underscore}
\usepackage{amsthm}

% A new command to mark not done places
\newcommand{\todo}[1][]{{\color{red}TODO\ #1}}

\newcommand{\abbr}{\textit{англ.}\ }

\subtitle{Курс «Параллельное программирование»}
\subject{Лекция}
\author[Евгений Юлюгин]{Евгений Юлюгин \\ \small{\href{mailto:yulyugin@gmail.com}{yulyugin@gmail.com}}}
\date{\today}
\pgfdeclareimage[height=0.5cm]{mipt-logo}{../common/mipt.png}
\logo{\pgfuseimage{mipt-logo}}

\typeout{Copyright 2014 Evgeny Yulyugin}

\usetheme{Berlin}
\setbeamertemplate{navigation symbols}{}%remove navigation symbols


\title{Области применения многопроцессорных систем. Примеры многопроцессорных и распределенных систем.}

\begin{document}

\begin{frame}
\titlepage
\end{frame}

\section*{Обзор}

\begin{frame}{На этой лекции}
\tableofcontents
\end{frame}

\section{Область применения}

\begin{frame}{Область применения}
\begin{itemize}
    \item Моделирование и анализ транспортных потоков,
    \item Разработка лекарственных препаратов,
    \item Прогнозирование погоды,
    \item Поисковые системы,
    \item Распознование изображений,
    \item Банковский сектор,
    \item Наука.
\end{itemize}
\end{frame}

\begin{frame}{Кто использует МВС?}
\begin{figure}[htpb]
\pause
\begin{minipage}[htpb]{0.3\textwidth}
\center{\includegraphics[width=\textwidth]{google-logo}}
\end{minipage}
\pause
\hspace*{0.05\textwidth}
\begin{minipage}[htpb]{0.25\textwidth}
\center{\includegraphics[width=\textwidth]{intel-logo}}
\end{minipage}
\pause
\vfill
\begin{minipage}[htpb]{0.3\textwidth}
\center{\includegraphics[width=\textwidth]{nasa-logo}}
\end{minipage}
\pause
\begin{minipage}[htpb]{0.3\textwidth}
\center{\includegraphics[width=\textwidth]{sberbank-logo}}
\end{minipage}
\pause
\begin{minipage}[htpb]{0.3\textwidth}
\center{\includegraphics[width=\textwidth]{common/mipt}}
\end{minipage}
\end{figure}
\end{frame}

\section{Типы МВС}

\begin{frame}{Типы МВС}
Наиболее распространенные типы многопроцессорных вычислительных систем:
\begin{itemize}
    \item Системы высокой надежности,
    \item Системы для высокопроизводительных вычислений,
    \item Многопоточные системы.
\end{itemize}
\end{frame}

\section{Требования}

\begin{frame}{Требование к системе}
\begin{itemize}
    \item Производительность,
    \item Энергопотребление,
    \item Время простоя,
    \item Надежность и отказоустойчивость,
    \item Масштабируемость.
\end{itemize}
\end{frame}

\section{Измерение производительности}

\begin{frame}{MIPS}
\end{frame}

\begin{frame}{FLOPS}
\end{frame}

\begin{frame}{Top-500}
\end{frame}

\section{Примеры}

\section{Литература}

\begin{frame}[allowframebreaks]{Рекомендуемая литература}
\begin{thebibliography}{99}
    \bibitem{} \textit{Богданов~А.В., Корхов~В.В., Мареев~В.В., Станкова~Е.Н.}
    Архитектуры и топологии многопроцессорных вычислительных систем. --- М.:
    ИНТУТ.РУ, 2004 --- 176~с. ISBN 5-9556-0018-3.
    \bibitem{} \textit{Таненбаум~Э.} Архитектура компьютера. --- 5-е изд. ---
    СПб.~Питер, 2007 --- 844~c. ISBN 5-469-01274-3.
    \bibitem{} \textit{John L. Hennessy, David A. Patterson} Computer
    Architecture: A Quantitative Approach --- 4th ed. --- 2007 --- 676~p.
\end{thebibliography}
\end{frame}

\section*{Конец}
% The final "thank you" frame

\begin{frame}{Задания}
\end{frame}

\begin{frame}{На следующей лекции}
\end{frame}

\begin{frame}

{\huge{Спасибо за внимание!}\par}

\vfill

\tiny{\textit{Замечание}: все торговые марки и логотипы, использованные в данном материале, являются собственностью их владельцев. Представленная здесь точка зрения отражает личное мнение автора, не выступающего от лица какой-либо организации.}

\end{frame}

\end{document}
