% Compile with XeLaTeX, TeXLive 2013 or more recent
\documentclass{beamer}

% Base packages
\usepackage{fontspec}
\defaultfontfeatures{Scale=MatchLowercase, Mapping=tex-text}

\usepackage{amsfonts}
\usepackage{amsmath}
\usepackage{longtable}
\usepackage{csquotes}
\usepackage{standalone}

\usepackage{graphicx}
\graphicspath{{./images/}}

\usepackage{tikz}
\usetikzlibrary{shapes, calc, arrows, decorations, patterns, chains, fit, petri}
\usetikzlibrary{backgrounds, positioning, decorations.pathreplacing}

\newcommand{\inputpicture}[1]{\input{../pictures/#1}}

\usepackage{tabularx}
\usepackage{multirow}

\usepackage{listings}
\lstset{language=C, basicstyle=\ttfamily, breaklines=true, keepspaces=true, keywordstyle=\color{blue}}

% Setup fonts
\newfontfamily\russianfont{CMU Serif}
\setromanfont{CMU Serif}
\setsansfont{CMU Sans Serif}
%\setmonofont{CMU Typewriter Text}

% Be able to insert hyperlinks
\usepackage{hyperref}
\hypersetup{colorlinks=true, linkcolor=black, filecolor=black, citecolor=black, urlcolor=blue , pdfauthor=Evgeny Yulyugin <yulyugin@gmail.com>, pdftitle=Параллельное программирование}
% \usepackage{url}

% Misc optional packages
\usepackage{underscore}
\usepackage{amsthm}

% A new command to mark not done places
\newcommand{\todo}[1][]{{\color{red}TODO\ #1}}

\newcommand{\abbr}{\textit{англ.}\ }

\newif\ifmipt
\newif\ifsbertech
\input{target}

\ifmipt
\subtitle{Курс <<Параллельное программирование>>}
\fi
\ifsbertech
\subtitle{Курс <<Инфраструктура многопроцессорных систем>>}
\fi
\subject{Лекция}
\author[Евгений Юлюгин]{Евгений Юлюгин \texorpdfstring{\\}{Lg} \small{\href{mailto:yulyugin@gmail.com}{yulyugin@gmail.com}}}
\date{\today}
\pgfdeclareimage[height=0.5cm]{mipt-logo}{common/mipt.png}
\logo{\pgfuseimage{mipt-logo}}

\typeout{Copyright 2014 Evgeny Yulyugin}

\usetheme{Berlin}
\setbeamertemplate{navigation symbols}{}%remove navigation symbols


\title{GPU}

\begin{document}

\begin{frame}
\titlepage
\end{frame}

\section*{Обзор}

\begin{frame}{На прошлой лекции}
\end{frame}

\begin{frame}{На этой лекции}
\tableofcontents
\end{frame}

\section{GPU}

\begin{frame}{GPU}
GPU (\abbr Graphics Processing Unit)
\end{frame}

\begin{frame}{GPU vs. CPU}
Отличия GPU и CPU:
\begin{itemize}
    \item GPU дополняет CPU $\Rightarrow$ GPU может выполнять некоторые задачи,
    присущие CPU, недостаточно эффективно или не выполнять их вообще;
    \item GPU совместимы на уровне API (OpenGL, Microsoft DirectX);
    \item Отсутствие бинарной обратной совместимости увеличивает скорость
    развития GPU архитектуры;
    \item GPU способны одновеременно выполнять множество потоков. Каждая точка
    может быть нарисована независимо;
    \item Временная локальность не применима к выполняемым GPU задачам.
\end{itemize}
\end{frame}

\section*{Литература}

\begin{frame}{Рекомендуемая литература}
\begin{thebibliography}{99}
    \bibitem{} \textit{Д.~Паттерсон, Дж.~Хеннесси}. Архитектура компьютера и
    проектирование компьютерных систем. 4-е издание --- СПб.~Питер, 2012 ---
    784~с. ISBN: 978-5-459-00291-1.
\end{thebibliography}
\end{frame}

\section*{Конец}
% The final "thank you" frame

\begin{frame}{Задания}
\end{frame}

\begin{frame}{На следующей лекции}
\end{frame}

\begin{frame}

{\huge{Спасибо за внимание!}\par}

\vfill

\tiny{\textit{Замечание}: все торговые марки и логотипы, использованные в данном материале, являются собственностью их владельцев. Представленная здесь точка зрения отражает личное мнение автора, не выступающего от лица какой-либо организации.}

\end{frame}

\end{document}
