% Compile with XeLaTeX, TeXLive 2013 or more recent
\documentclass{beamer}

% Base packages
\usepackage{fontspec}
\defaultfontfeatures{Scale=MatchLowercase, Mapping=tex-text}

\usepackage{amsfonts}
\usepackage{amsmath}
\usepackage{longtable}
\usepackage{csquotes}
\usepackage{standalone}

\usepackage{graphicx}
\graphicspath{{./images/}}

\usepackage{tikz}
\usetikzlibrary{shapes, calc, arrows, decorations, patterns, chains, fit, petri}
\usetikzlibrary{backgrounds, positioning, decorations.pathreplacing}

\newcommand{\inputpicture}[1]{\input{../pictures/#1}}

\usepackage{tabularx}
\usepackage{multirow}

\usepackage{listings}
\lstset{language=C, basicstyle=\ttfamily, breaklines=true, keepspaces=true, keywordstyle=\color{blue}}

% Setup fonts
\newfontfamily\russianfont{CMU Serif}
\setromanfont{CMU Serif}
\setsansfont{CMU Sans Serif}
%\setmonofont{CMU Typewriter Text}

% Be able to insert hyperlinks
\usepackage{hyperref}
\hypersetup{colorlinks=true, linkcolor=black, filecolor=black, citecolor=black, urlcolor=blue , pdfauthor=Evgeny Yulyugin <yulyugin@gmail.com>, pdftitle=Параллельное программирование}
% \usepackage{url}

% Misc optional packages
\usepackage{underscore}
\usepackage{amsthm}

% A new command to mark not done places
\newcommand{\todo}[1][]{{\color{red}TODO\ #1}}

\newcommand{\abbr}{\textit{англ.}\ }

\newif\ifmipt
\newif\ifsbertech
\input{target}

\ifmipt
\subtitle{Курс <<Параллельное программирование>>}
\fi
\ifsbertech
\subtitle{Курс <<Инфраструктура многопроцессорных систем>>}
\fi
\subject{Лекция}
\author[Евгений Юлюгин]{Евгений Юлюгин \texorpdfstring{\\}{Lg} \small{\href{mailto:yulyugin@gmail.com}{yulyugin@gmail.com}}}
\date{\today}
\pgfdeclareimage[height=0.5cm]{mipt-logo}{common/mipt.png}
\logo{\pgfuseimage{mipt-logo}}

\typeout{Copyright 2014 Evgeny Yulyugin}

\usetheme{Berlin}
\setbeamertemplate{navigation symbols}{}%remove navigation symbols


\title{Разработка многопоточных приложений на Java.}

\begin{document}

\begin{frame}
\titlepage
\end{frame}

\section*{Обзор}

\begin{frame}{На прошлой лекции}
\begin{itemize}
\ifsbertech
    \item Атомарные операции,
    \item Состояние гонки,
    \item Примитивы синхронизации,
    \item Программная реализация,
    \item Популярные ошибки.
\fi
\end{itemize}
\end{frame}

\begin{frame}{На этой лекции}
\tableofcontents
\end{frame}

\section{Создание множества потоков}

\begin{frame}[fragile]{Runnable интерфейс}
\begin{lstlisting}[basicstyle=\footnotesize, language=Java]
public class HelloRunnable implements Runnable {

    public void run() {
        System.out.println("Hello from a thread!");
    }

    public static void main(String args[]) {
        (new Thread(new HelloRunnable())).start();
    }

}
\end{lstlisting}
\end{frame}

\begin{frame}[fragile]{Наследование Thread}
\begin{lstlisting}[basicstyle=\footnotesize, language=Java]
public class HelloThread extends Thread {

    public void run() {
        System.out.println("Hello from a thread!");
    }

    public static void main(String args[]) {
        (new HelloThread()).start();
    }

}
\end{lstlisting}
\end{frame}

\section{Атомарные типы}

\begin{frame}{Атомарные типы}
Пакет java.util.concurrent.atomic
\vfill
\begin{itemize}
    \item AtomicBoolean,
    \item AtomicInteger,
    \item AtomicLong,
    \item AtomicReference<Type>.
\end{itemize}
\vfill
Операции:
\begin{itemize}
    \item Type get();
    \item void set(Type newValue);
    \item boolean compareAndSet(Type expect, Type update);
\end{itemize}
\end{frame}

\begin{frame}[fragile]{Атомарные операции}
Метод \texttt{compareAndSet} позволяет реализовывать другие операции.
\vfill
Пример из \texttt{java.util.concurrent.atomic.AtomicInteger}
\begin{lstlisting}[language=Java]
public final int getAndDecrement() {
        for (;;) {
            int current = get();
            int next = current - 1;
            if (compareAndSet(current, next))
                return current;
        }
    }
\end{lstlisting}
\end{frame}

\section{Примитивы синхронизации}

\begin{frame}{Synchronized}
Synchronized блоки помечаются специальным ключевым словом \textcolor{blue}{\texttt{synchronyzed}}.
\vfill
Только один потом может находиться внутри synchronized блока.
\vfill
Ключевое слово \textcolor{blue}{\texttt{synchronyzed}} может быть использовано для:
\begin{itemize}
    \item Instance методы,
    \item Static методы,
    \item Блоки внутри Istance методов,
    \item Блоки внутри Static методов.
\end{itemize}
\end{frame}

\begin{frame}[fragile]{Synchronized instance method}
Synchronized on instance (object) owning the method.
\vfill
\begin{lstlisting}[language=Java]
public synchronized void add(int val) {
    this.count += val;
}
\end{lstlisting}
\vfill
\begin{lstlisting}[language=Java]
public void add(int val) {
    synchronized(this) {
        this.count += val;
    }
}
\end{lstlisting}
\vfill
One thread per instance.
\end{frame}

\begin{frame}[fragile]{Synchronized static method}
\begin{lstlisting}[language=Java]
public static synchronized void add(int val) {
    this.count += val;
}
\end{lstlisting}
\vfill
\begin{lstlisting}[language=Java]
public static void add(int val) {
    synchronized(MyClass.class) {
        this.count += val;
    }
}
\end{lstlisting}
\vfill
One thread per class object.
\end{frame}

\begin{frame}{Захват монитора}
\begin{itemize}
    \item \texttt{\textcolor{blue}{void} wait();}
    \item \texttt{\textcolor{blue}{void} wait(\textcolor{blue}{long} timeout);}
    \item \texttt{\textcolor{blue}{void} notify();}
    \item \texttt{\textcolor{blue}{void} notifyAll();}
\end{itemize}
\end{frame}

\begin{frame}[fragile]{Пример}
\begin{lstlisting}[basicstyle=\tiny, language=Java]
public class ThreadA {
    public static void main(String[] args){
        ThreadB b = new ThreadB();
        b.start();
        synchronized(b){
            try{
                System.out.println("Waiting for b");
                b.wait();
            }catch(InterruptedException e){
                e.printStackTrace();
            }
            System.out.println("Total is: " + b.total);
        }
    }
}
class ThreadB extends Thread{
    int total;
    @Override
    public void run(){
        synchronized(this){
            for(int i=0; i<100 ; i++){
                total += i;
            }
            notify();
        }
    }
}
\end{lstlisting}
\end{frame}

\begin{frame}{Semaphore}
\begin{itemize}
    \item Класс \texttt{java.util.concurrent.Semaphore};
    \item Ограничивает одновременный доступ к ресурсам;
    \item В отличие от \textcolor{blue}{\texttt{synchronyzed}} блока, одновременно могут работать несколько потоков (не более заданного числа);
    \item Операции:
    \begin{itemize}
        \item \texttt{\textcolor{blue}{void} acquire()};
        \item \texttt{\textcolor{blue}{void} release()};
    \end{itemize}
\end{itemize}
\end{frame}

\begin{frame}[fragile]{Semaphore example}
\begin{lstlisting}[language=Java]
Semaphore semaphore = new Semaphore(4);

semaphore.acquire();

try {
    /* Up to four threads may run
       this code concurrently */
} finally {
    semaphore.release();
}
\end{lstlisting}
\end{frame}

\begin{frame}{Lock}
Операции:
\begin{itemize}
    \item \texttt{\textcolor{blue}{void} lock();}
    \item \texttt{\textcolor{blue}{boolean} tryLock();}
    \item \texttt{\textcolor{blue}{boolean} tryLock(\textcolor{blue}{long} time, TimeUnit unit);}
    \item \texttt{\textcolor{blue}{void} unlock();}
    \item \texttt{Condition newCondition();}
\end{itemize}
\end{frame}

\begin{frame}{CountDownLatch}
Класс \texttt{java.util.concurrent.CountDownLatch}.
\vfill
Обеспечивает точку синхронизации между N потоками.
\vfill
Операции:
\begin{itemize}
    \item \texttt{\textcolor{blue}{void} await();}
    \item \texttt{\textcolor{blue}{void} countDown();}
\end{itemize}
\end{frame}

\begin{frame}[fragile]{CountDownLatch пример}
\begin{lstlisting}[language=Java]
CountDownLatch latch = new CountDownLatch(4);

/* This call blocks util latch.countDown()
 * is called at least 4 times. */
latch.await();
\end{lstlisting}
\end{frame}

\section*{Конец}

\begin{frame}[allowframebreaks]{Рекомендуемая литература}
\begin{thebibliography}{99}
    \bibitem{} Обзор java.util.concurrent.
    \url{http://habrahabr.ru/company/luxoft/blog/157273}.
    \bibitem{} The Java Tutorials.
    \href{http://docs.oracle.com/javase/tutorial/essential/concurrency/index.html}{Lesson: Concurrency.}
    \bibitem{} \href{http://docs.oracle.com/javase/7/docs/api/java/util/concurrent/package-summary.html}{Package java.util.concurrent}.
\end{thebibliography}
\end{frame}

\begin{frame}{Задания}
\end{frame}

\begin{frame}{На следующей лекции}
\begin{itemize}
    \item Классификации параллельных вычислительных систем:
    \begin{itemize}
        \item Классификация Флинна,
        \item Классификация Хокни,
    \end{itemize}
    \item Суперскалярная архитектура,
    \item VLIW арфитектура.
\end{itemize}
\end{frame}

\begin{frame}

{\huge{Спасибо за внимание!}\par}

\vfill

\tiny{\textit{Замечание}: все торговые марки и логотипы, использованные в данном материале, являются собственностью их владельцев. Представленная здесь точка зрения отражает личное мнение автора, не выступающего от лица какой-либо организации.}

\end{frame}

\end{document}
