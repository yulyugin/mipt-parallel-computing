% Compile with XeLaTeX, TeXLive 2013 or more recent
\documentclass{beamer}

% Base packages
\usepackage{fontspec}
\defaultfontfeatures{Scale=MatchLowercase, Mapping=tex-text}

\usepackage{amsfonts}
\usepackage{amsmath}
\usepackage{longtable}
\usepackage{csquotes}
\usepackage{standalone}

\usepackage{graphicx}
\graphicspath{{./images/}}

\usepackage{tikz}
\usetikzlibrary{shapes, calc, arrows, decorations, patterns, chains, fit, petri}
\usetikzlibrary{backgrounds, positioning, decorations.pathreplacing}

\newcommand{\inputpicture}[1]{\input{../pictures/#1}}

\usepackage{tabularx}
\usepackage{multirow}

\usepackage{listings}
\lstset{language=C, basicstyle=\ttfamily, breaklines=true, keepspaces=true, keywordstyle=\color{blue}}

% Setup fonts
\newfontfamily\russianfont{CMU Serif}
\setromanfont{CMU Serif}
\setsansfont{CMU Sans Serif}
%\setmonofont{CMU Typewriter Text}

% Be able to insert hyperlinks
\usepackage{hyperref}
\hypersetup{colorlinks=true, linkcolor=black, filecolor=black, citecolor=black, urlcolor=blue , pdfauthor=Evgeny Yulyugin <yulyugin@gmail.com>, pdftitle=Параллельное программирование}
% \usepackage{url}

% Misc optional packages
\usepackage{underscore}
\usepackage{amsthm}

% A new command to mark not done places
\newcommand{\todo}[1][]{{\color{red}TODO\ #1}}

\newcommand{\abbr}{\textit{англ.}\ }

\newif\ifmipt
\newif\ifsbertech
\input{target}

\ifmipt
\subtitle{Курс <<Параллельное программирование>>}
\fi
\ifsbertech
\subtitle{Курс <<Инфраструктура многопроцессорных систем>>}
\fi
\subject{Лекция}
\author[Евгений Юлюгин]{Евгений Юлюгин \texorpdfstring{\\}{Lg} \small{\href{mailto:yulyugin@gmail.com}{yulyugin@gmail.com}}}
\date{\today}
\pgfdeclareimage[height=0.5cm]{mipt-logo}{common/mipt.png}
\logo{\pgfuseimage{mipt-logo}}

\typeout{Copyright 2014 Evgeny Yulyugin}

\usetheme{Berlin}
\setbeamertemplate{navigation symbols}{}%remove navigation symbols


\title{Коллективные операции в MPI}

\begin{document}

\begin{frame}
\titlepage
\end{frame}

\section*{Обзор}

\begin{frame}{На прошлой лекции}
\begin{itemize}
\ifmipt
    \item Классификация Флинна.
    \item Классификация Хокни.
    \item Общая и расределенная память.
\fi
\end{itemize}
\end{frame}

\begin{frame}{На этой лекции}
\tableofcontents
\end{frame} 

% Use [fragile] option to insert listings
% \begin{frame}[fragile]

\section{Синхронизация}

\begin{frame}[fragile]{MPI_Barrier}

Блокирует работу вызвавшего ее процесса до тех пор, пока все другие процессы группы также не вызовут эту функцию.

\begin{lstlisting}
int MPI_Barrier(MPI_Comm comm);
\end{lstlisting}

Все процессы преодолевают барьер одновременно.

\end{frame}

\begin{frame}[fragile]{Пример}

\begin{lstlisting}
double timestamp0, timestamp1;

MPI_Barrier(MPI_COMM_WORLD);
if (rank == 0)
  timestamp0 = MPI_Wtime();

// ...

MPI_Barrier(MPI_COMM_WORLD);
if (rank == 0)
  timestamp1 = MPI_Wtime();
\end{lstlisting}

\end{frame}

\section{Широковещательная рассылка}

\begin{frame}{Неявная синхронизация}

Неявную синхронизацию процессов выполняет любая коллективная функция.

\end{frame}

\begin{frame}[fragile]{MPI_Bcast}

Рассылка сообщения из \texttt{buffer} процесса \texttt{root} всем процессам коммуникатора.

\begin{lstlisting}
int MPI_Bcast(void *buffer, int count, MPI_Datatype datatype, int root, MPI_Comm comm);
\end{lstlisting}

\end{frame}

\begin{frame}{MPI_Bcast}

\begin{figure}
    \centering
    \begin{tikzpicture}
        \node[] (root) {Процесс root};
        \node[below = 2.5cm of root] (all) {Все процессы};

        \node[draw, fill=orange, rectangle, right= 3.5cm of root, minimum width= 2cm, minimum height= 0.5cm] (memroot) {};
        \draw[->] ([yshift= 0.3cm] memroot.north west) -- (memroot.north west);
        \node[above left= 0cm and -0.1cm of memroot.north] {\scriptsize{\texttt{buffer}}};
        \draw[<->] ([yshift= -0.2cm] memroot.south west) -- ([yshift= -0.2cm] memroot.south east);
        \node[fill=white, below= 0cm of memroot.south] {\scriptsize{\texttt{count}}};

        \node[draw, fill=orange, rectangle, right= 1cm of all, minimum width= 2cm, minimum height= 0.5cm] (proc0) {};
        \draw[->] ([yshift= 0.3cm] proc0.north west) -- (proc0.north west);
        \node[above left= 0cm and -0.1cm of proc0.north] {\scriptsize{\texttt{buffer}}};
        \draw[<->] ([yshift= -0.2cm] proc0.south west) -- ([yshift= -0.2cm] proc0.south east);
        \node[fill=white, below= 0cm of proc0.south] {\scriptsize{\texttt{count}}};
        \node[below= 0.3cm of proc0.south] {\scriptsize{процесс 0}};
        \node[right= 0cm of proc0, minimum width= 0cm, minimum height= 0.5cm, align=center] (etc0) {...};

        \node[draw, fill=orange, rectangle, right= 0cm of etc0, minimum width= 2cm, minimum height= 0.5cm] (proc1) {};
        \draw[->] ([yshift= 0.3cm] proc1.north west) -- (proc1.north west);
        \node[above left= 0cm and -0.1cm of proc1.north] {\scriptsize{\texttt{buffer}}};
        \draw[<->] ([yshift= -0.2cm] proc1.south west) -- ([yshift= -0.2cm] proc1.south east);
        \node[fill=white, below= 0cm of proc1.south] {\scriptsize{\texttt{count}}};
        \node[below= 0.3cm of proc1.south] {\scriptsize{процесс root}};
        \node[right= 0cm of proc1, minimum width= 0.2cm, minimum height= 0.5cm, align=center] (etc1) {...};

        \node[draw, fill=orange, rectangle, right= 0cm of etc1, minimum width= 2cm, minimum height= 0.5cm] (procN) {};
        \draw[->] ([yshift= 0.3cm] procN.north west) -- (procN.north west);
        \node[above left= 0cm and -0.1cm of procN.north] {\scriptsize{\texttt{buffer}}};
        \draw[<->] ([yshift= -0.2cm] procN.south west) -- ([yshift= -0.2cm] procN.south east);
        \node[fill=white, below= 0cm of procN.south] {\scriptsize{\texttt{count}}};
        \node[below= 0.3cm of procN.south] {\scriptsize{процесс n-1}};

        \path[] (memroot) -- (proc0) node[draw, rotate=-135, pos=0.5, single arrow, minimum height=1.5cm] {};
        \path[] (memroot) -- (proc1) node[draw, rotate=-90, pos=0.5, single arrow, minimum height=1.5cm] {};
        \path[] (memroot) -- (procN) node[draw, rotate=-45, pos=0.5, single arrow, minimum height=1.5cm] {};
    \end{tikzpicture}
\end{figure}

\end{frame}

\begin{frame}[fragile]{Пример}

\begin{lstlisting}
int n;
if (rank == 0) {
  printf("Enter parameter N:\n");
  scanf("%d", &n);
}
MPI_Bcast(&n, 1, MPI_INT, 0, MPI_COMM_WORLD);
\end{lstlisting}

\end{frame}

\section{Распределение данных}

\begin{frame}[fragile]{MPI_Scatter}

\begin{lstlisting}
int MPI_Scatter(const void *sendbuf, int sendcount, MPI_Datatype sendtype, void *recvbuf, int recvcount, MPI_Datatype recvtype, int root, MPI_Comm comm);
\end{lstlisting}

\end{frame}

\begin{frame}{MPI_Scatter}

\begin{figure}[htp]
    \centering
    \begin{tikzpicture}
        \node[] (root) {Процесс root};
        \node[below = 2cm of root] (all) {Все процессы};

        \node[draw, fill=orange, rectangle, right= 1cm of root, minimum width= 2cm, minimum height= 0.5cm] (memroot0) {};
        \draw[->] ([yshift= 0.3cm] memroot0.north west) -- (memroot0.north west);
        \node[above left= 0cm and -0.2cm of memroot0.north] {\scriptsize{\texttt{sendbuf}}};
        \draw[<->] ([yshift= -0.2cm] memroot0.south west) -- ([yshift= -0.2cm] memroot0.south east);
        \node[fill=white, below= 0cm of memroot0.south] {\scriptsize{\texttt{sendcount}}};

        \node[draw, fill=orange, rectangle, right= 0cm of memroot0, minimum width= 2cm, minimum height= 0.5cm] (memroot1) {};
        \draw[<->] ([yshift= -0.2cm] memroot1.south west) -- ([yshift= -0.2cm] memroot1.south east);
        \node[fill=white, below= 0cm of memroot1.south] {\scriptsize{\texttt{sendcount}}};
        \node[draw, rectangle, right= 0cm of memroot1, minimum width= 0.2cm, minimum height= 0.5cm, align=center] (etcroot) {...};

        \node[draw, fill=orange, rectangle, right= 0cm of etcroot, minimum width= 2cm, minimum height= 0.5cm] (memrootN) {};
        \draw[<->] ([yshift= -0.2cm] memrootN.south west) -- ([yshift= -0.2cm] memrootN.south east);
        \node[fill=white, below= 0cm of memrootN.south] {\scriptsize{\texttt{sendcount}}};

        \node[draw, fill=orange, rectangle, right= 0.75cm of all, minimum width= 2cm, minimum height= 0.5cm] (memall0) {};
        \draw[->] ([yshift= 0.3cm] memall0.north west) -- (memall0.north west);
        \node[above left= 0cm and -0.2cm of memall0.north] {\scriptsize{\texttt{recvbuf}}};
        \draw[<->] ([yshift= -0.2cm] memall0.south west) -- ([yshift= -0.2cm] memall0.south east);
        \node[fill=white, below= 0cm of memall0.south] {\scriptsize{\texttt{recvcount}}};

        \node[draw, fill=orange, rectangle, right= 0.5cm of memall0, minimum width= 2cm, minimum height= 0.5cm] (memall1) {};
        \draw[->] ([yshift= 0.3cm] memall1.north west) -- (memall1.north west);
        \node[above left= 0cm and -0.2cm of memall1.north] {\scriptsize{\texttt{recvbuf}}};
        \draw[<->] ([yshift= -0.2cm] memall1.south west) -- ([yshift= -0.2cm] memall1.south east);
        \node[fill=white, below= 0cm of memall1.south] {\scriptsize{\texttt{recvcount}}};
        \node[right= 0cm of memall1, minimum width= 0.2cm, minimum height= 0.5cm, align=center] (etcall) {...};

        \node[draw, fill=orange, rectangle, right= 0cm of etcall, minimum width= 2cm, minimum height= 0.5cm] (memallN) {};
        \draw[->] ([yshift= 0.3cm] memallN.north west) -- (memallN.north west);
        \node[above left= 0cm and -0.2cm of memallN.north] {\scriptsize{\texttt{recvbuf}}};
        \draw[<->] ([yshift= -0.2cm] memallN.south west) -- ([yshift= -0.2cm] memallN.south east);
        \node[fill=white, below= 0cm of memallN.south] {\scriptsize{\texttt{recvcount}}};

        \path[] (memroot0) -- (memall0) node[draw, rotate=-90, pos=0.45, single arrow, minimum height=1.2cm] {};
        \path[] (memroot1) -- (memall1) node[draw, rotate=-90, pos=0.45, single arrow, minimum height=1.2cm] {};
        \path[] (memrootN) -- (memallN) node[draw, rotate=-90, pos=0.45, single arrow, minimum height=1.2cm] {};
    \end{tikzpicture}
\end{figure}

\end{frame}

\begin{frame}[fragile]{Пример}

\begin{lstlisting}
MPI_Comm comm;
int  rbuf[100], size;
int  root, *array;
// ...
MPI_Comm_size(comm, &size);
array = (int *) malloc(size * 100 * sizeof(int));
// ...
MPI_Scatter(array, 100, MPI_INT, rbuf, 100, MPI_INT, root, comm);
\end{lstlisting}

\end{frame}

\begin{frame}[fragile]{MPI_Scatterv}

\begin{lstlisting}
int MPI_Scatterv(const void *sendbuf, const int *sendcounts, const int *displs, MPI_Datatype sendtype, void *recvbuf, int recvcount, MPI_Datatype recvtype, int root, MPI_Comm comm);
\end{lstlisting}

\end{frame}

\begin{frame}{MPI_Scatterv}

\begin{figure}[htp]
    \centering
    \begin{tikzpicture}
        \node[] (root) {Процесс root};
        \node[below = 3cm of root] (all) {Все процессы};

        \node[draw, fill=orange, rectangle, right= 0.5cm of root, minimum width= 2cm, minimum height= 0.5cm] (memroot0) {};
        \draw[->] ([yshift= 0.3cm] memroot0.north west) -- (memroot0.north west);
        \node[above left= 0cm and -0.2cm of memroot0.north] {\scriptsize{\texttt{sendbuf}}};
        \node[draw, rectangle, right= 0cm of memroot0, minimum width= 0.2cm, minimum height= 0.5cm, align=center] (etcroot0) {...};

        \node[draw, fill=brown, rectangle, right= 0cm of etcroot0, minimum width= 2.6cm, minimum height= 0.5cm] (memroot1) {};
        \draw[<->] ([yshift= -0.2cm] memroot1.south west) -- ([yshift= -0.2cm] memroot1.south east);
        \node[fill=white, below= 0cm of memroot1.south] {\scriptsize{\texttt{sendcounts[i]}}};
        \node[draw, rectangle, right= 0cm of memroot1, minimum width= 0.2cm, minimum height= 0.5cm, align=center] (etcroot1) {...};

        \node[draw, fill=yellow, rectangle, right= 0cm of etcroot1, minimum width= 2cm, minimum height= 0.5cm] (memrootN) {};

        \draw[-] ([yshift= -0.5cm] memroot0.south west) -- (memroot0.south west);
        \draw[-] ([yshift= -0.5cm] memroot1.south west) -- (memroot1.south west);
        \draw[<->] ([yshift= -0.4cm] memroot0.south west) -- ([yshift= -0.4cm] memroot1.south west);
        \node[fill=white, below right= 0.15cm and -0.5cm of memroot0.south] {\scriptsize{\texttt{displs[i]}}};

        \node[draw, fill=orange, rectangle, right= 0.5cm of all, minimum width= 2cm, minimum height= 0.5cm] (memall0) {};
        \draw[->] ([yshift= 0.3cm] memall0.north west) -- (memall0.north west);
        \node[above left= 0cm and -0.2cm of memall0.north] {\scriptsize{\texttt{recvbuf}}};
        \draw[<->] ([yshift= -0.2cm] memall0.south west) -- ([yshift= -0.2cm] memall0.south east);
        \node[fill=white, below= 0cm of memall0.south] {\scriptsize{\texttt{recvcount}}};
        \node[right= 0cm of memall0, minimum width= 0.2cm, minimum height= 0.5cm, align=center] (etcall0) {...};

        \node[draw, fill=brown, rectangle, right= 0cm of etcall0, minimum width= 2.6cm, minimum height= 0.5cm] (memall1) {};
        \draw[->] ([yshift= 0.3cm] memall1.north west) -- (memall1.north west);
        \node[above left= 0cm and 0.05cm of memall1.north] {\scriptsize{\texttt{recvbuf}}};
        \draw[<->] ([yshift= -0.2cm] memall1.south west) -- ([yshift= -0.2cm] memall1.south east);
        \node[fill=white, below= 0cm of memall1.south] {\scriptsize{\texttt{recvcount}}};
        \node[right= 0cm of memall1, minimum width= 0.2cm, minimum height= 0.5cm, align=center] (etcall1) {...};

        \node[draw, fill=yellow, rectangle, right= 0cm of etcall1, minimum width= 2cm, minimum height= 0.5cm] (memallN) {};
        \draw[->] ([yshift= 0.3cm] memallN.north west) -- (memallN.north west);
        \node[above left= 0cm and -0.2cm of memallN.north] {\scriptsize{\texttt{recvbuf}}};
        \draw[<->] ([yshift= -0.2cm] memallN.south west) -- ([yshift= -0.2cm] memallN.south east);
        \node[fill=white, below= 0cm of memallN.south] {\scriptsize{\texttt{recvcount}}};

        \path[] (memroot0) -- (memall0) node[draw, rotate=-90, pos=0.5, single arrow, minimum height=1.5cm] {};
        \path[] (memroot1) -- (memall1) node[draw, rotate=-90, pos=0.5, single arrow, minimum height=1.5cm] {};
        \path[] (memrootN) -- (memallN) node[draw, rotate=-90, pos=0.5, single arrow, minimum height=1.5cm] {};
    \end{tikzpicture}
\end{figure}

\end{frame}

\section{Сбор данных}

\begin{frame}[fragile]{MPI_Gather}

\begin{lstlisting}
int MPI_Gather(const void *sendbuf, int sendcount, MPI_Datatype sendtype, void *recvbuf, int recvcount, MPI_Datatype recvtype, int root, MPI_Comm comm);
\end{lstlisting}

\end{frame}

\begin{frame}{MPI_Gather}

\begin{figure}[htp]
    \centering
    \begin{tikzpicture}
        \node[] (root) {Процесс root};
        \node[below = 2cm of root] (all) {Все процессы};

        \node[draw, fill=orange, rectangle, right= 1cm of root, minimum width= 2cm, minimum height= 0.5cm] (memroot0) {};
        \draw[->] ([yshift= 0.3cm] memroot0.north west) -- (memroot0.north west);
        \node[above left= 0cm and -0.2cm of memroot0.north] {\scriptsize{\texttt{recvbuf}}};
        \draw[<->] ([yshift= -0.2cm] memroot0.south west) -- ([yshift= -0.2cm] memroot0.south east);
        \node[fill=white, below= 0cm of memroot0.south] {\scriptsize{\texttt{recvcount}}};

        \node[draw, fill=orange, rectangle, right= 0cm of memroot0, minimum width= 2cm, minimum height= 0.5cm] (memroot1) {};
        \draw[<->] ([yshift= -0.2cm] memroot1.south west) -- ([yshift= -0.2cm] memroot1.south east);
        \node[fill=white, below= 0cm of memroot1.south] {\scriptsize{\texttt{recvcount}}};
        \node[draw, rectangle, right= 0cm of memroot1, minimum width= 0.2cm, minimum height= 0.5cm, align=center] (etcroot) {...};

        \node[draw, fill=orange, rectangle, right= 0cm of etcroot, minimum width= 2cm, minimum height= 0.5cm] (memrootN) {};
        \draw[<->] ([yshift= -0.2cm] memrootN.south west) -- ([yshift= -0.2cm] memrootN.south east);
        \node[fill=white, below= 0cm of memrootN.south] {\scriptsize{\texttt{recvcount}}};

        \node[draw, fill=orange, rectangle, right= 0.75cm of all, minimum width= 2cm, minimum height= 0.5cm] (memall0) {};
        \draw[->] ([yshift= 0.3cm] memall0.north west) -- (memall0.north west);
        \node[above left= 0cm and -0.2cm of memall0.north] {\scriptsize{\texttt{sendbuf}}};
        \draw[<->] ([yshift= -0.2cm] memall0.south west) -- ([yshift= -0.2cm] memall0.south east);
        \node[fill=white, below= 0cm of memall0.south] {\scriptsize{\texttt{sendcount}}};

        \node[draw, fill=orange, rectangle, right= 0.5cm of memall0, minimum width= 2cm, minimum height= 0.5cm] (memall1) {};
        \draw[->] ([yshift= 0.3cm] memall1.north west) -- (memall1.north west);
        \node[above left= 0cm and -0.2cm of memall1.north] {\scriptsize{\texttt{sendbuf}}};
        \draw[<->] ([yshift= -0.2cm] memall1.south west) -- ([yshift= -0.2cm] memall1.south east);
        \node[fill=white, below= 0cm of memall1.south] {\scriptsize{\texttt{sendcount}}};
        \node[right= 0cm of memall1, minimum width= 0.2cm, minimum height= 0.5cm, align=center] (etcall) {...};

        \node[draw, fill=orange, rectangle, right= 0cm of etcall, minimum width= 2cm, minimum height= 0.5cm] (memallN) {};
        \draw[->] ([yshift= 0.3cm] memallN.north west) -- (memallN.north west);
        \node[above left= 0cm and -0.2cm of memallN.north] {\scriptsize{\texttt{sendbuf}}};
        \draw[<->] ([yshift= -0.2cm] memallN.south west) -- ([yshift= -0.2cm] memallN.south east);
        \node[fill=white, below= 0cm of memallN.south] {\scriptsize{\texttt{sendcount}}};

        \path[] (memroot0) -- (memall0) node[draw, rotate=90, pos=0.5, single arrow, minimum height=1.2cm] {};
        \path[] (memroot1) -- (memall1) node[draw, rotate=90, pos=0.5, single arrow, minimum height=1.2cm] {};
        \path[] (memrootN) -- (memallN) node[draw, rotate=90, pos=0.5, single arrow, minimum height=1.2cm] {};
    \end{tikzpicture}
\end{figure}

\end{frame}

\begin{frame}[fragile]{MPI_Gatherv}

\begin{lstlisting}
int MPI_Gatherv(void* sendbuf, int sendcount, MPI_Datatype sendtype, void* rbuf, int *recvcounts, int *displs, MPI_Datatype recvtype, int root, MPI_Comm comm);
\end{lstlisting}

\end{frame}

\begin{frame}{MPI_Gatherv}

\begin{figure}[htp]
    \centering
    \begin{tikzpicture}
        \node[] (root) {Процесс root};
        \node[below = 3cm of root] (all) {Все процессы};

        \node[draw, fill=orange, rectangle, right= 0.5cm of root, minimum width= 2cm, minimum height= 0.5cm] (memroot0) {};
        \draw[->] ([yshift= 0.3cm] memroot0.north west) -- (memroot0.north west);
        \node[above left= 0cm and -0.2cm of memroot0.north] {\scriptsize{\texttt{recvbuf}}};
        \node[draw, rectangle, right= 0cm of memroot0, minimum width= 0.2cm, minimum height= 0.5cm, align=center] (etcroot0) {...};

        \node[draw, fill=brown, rectangle, right= 0cm of etcroot0, minimum width= 2.6cm, minimum height= 0.5cm] (memroot1) {};
        \draw[<->] ([yshift= -0.2cm] memroot1.south west) -- ([yshift= -0.2cm] memroot1.south east);
        \node[fill=white, below= 0cm of memroot1.south] {\scriptsize{\texttt{recvcounts[i]}}};
        \node[draw, rectangle, right= 0cm of memroot1, minimum width= 0.2cm, minimum height= 0.5cm, align=center] (etcroot1) {...};

        \node[draw, fill=yellow, rectangle, right= 0cm of etcroot1, minimum width= 2cm, minimum height= 0.5cm] (memrootN) {};

        \draw[-] ([yshift= -0.5cm] memroot0.south west) -- (memroot0.south west);
        \draw[-] ([yshift= -0.5cm] memroot1.south west) -- (memroot1.south west);
        \draw[<->] ([yshift= -0.4cm] memroot0.south west) -- ([yshift= -0.4cm] memroot1.south west);
        \node[fill=white, below right= 0.15cm and -0.5cm of memroot0.south] {\scriptsize{\texttt{displs[i]}}};

        \node[draw, fill=orange, rectangle, right= 0.5cm of all, minimum width= 2cm, minimum height= 0.5cm] (memall0) {};
        \draw[->] ([yshift= 0.3cm] memall0.north west) -- (memall0.north west);
        \node[above left= 0cm and -0.2cm of memall0.north] {\scriptsize{\texttt{sendbuf}}};
        \draw[<->] ([yshift= -0.2cm] memall0.south west) -- ([yshift= -0.2cm] memall0.south east);
        \node[fill=white, below= 0cm of memall0.south] {\scriptsize{\texttt{sendcount}}};
        \node[right= 0cm of memall0, minimum width= 0.2cm, minimum height= 0.5cm, align=center] (etcall0) {...};

        \node[draw, fill=brown, rectangle, right= 0cm of etcall0, minimum width= 2.6cm, minimum height= 0.5cm] (memall1) {};
        \draw[->] ([yshift= 0.3cm] memall1.north west) -- (memall1.north west);
        \node[above left= 0cm and 0.05cm of memall1.north] {\scriptsize{\texttt{sendbuf}}};
        \draw[<->] ([yshift= -0.2cm] memall1.south west) -- ([yshift= -0.2cm] memall1.south east);
        \node[fill=white, below= 0cm of memall1.south] {\scriptsize{\texttt{sendcount}}};
        \node[right= 0cm of memall1, minimum width= 0.2cm, minimum height= 0.5cm, align=center] (etcall1) {...};

        \node[draw, fill=yellow, rectangle, right= 0cm of etcall1, minimum width= 2cm, minimum height= 0.5cm] (memallN) {};
        \draw[->] ([yshift= 0.3cm] memallN.north west) -- (memallN.north west);
        \node[above left= 0cm and -0.2cm of memallN.north] {\scriptsize{\texttt{sendbuf}}};
        \draw[<->] ([yshift= -0.2cm] memallN.south west) -- ([yshift= -0.2cm] memallN.south east);
        \node[fill=white, below= 0cm of memallN.south] {\scriptsize{\texttt{sendcount}}};

        \path[] (memroot0) -- (memall0) node[draw, rotate=90, pos=0.5, single arrow, minimum height=1.5cm] {};
        \path[] (memroot1) -- (memall1) node[draw, rotate=90, pos=0.5, single arrow, minimum height=1.5cm] {};
        \path[] (memrootN) -- (memallN) node[draw, rotate=90, pos=0.5, single arrow, minimum height=1.5cm] {};
    \end{tikzpicture}
\end{figure}

\end{frame}


\begin{frame}[fragile]{MPI_Allgather}

\begin{lstlisting}
int MPI_Allgather(void* sendbuf, int sendcount, MPI_Datatype sendtype, void* recvbuf, int recvcount, MPI_Datatype recvtype, MPI_Comm comm);
\end{lstlisting}

\end{frame}

\begin{frame}{MPI_Allgather}

\begin{figure}
    \centering
    \begin{tikzpicture}
        \node[] (proc0) {процесс 0};
        \node[below= 1cm of proc0] (proc1) {процесс 1};
        \node[below= 0.6cm of proc1] (etcproc) {...};
        \node[below= 0.6cm of etcproc] (procN) {процесс N};

        \node[draw, fill=orange, rectangle, right= 0.5cm of proc0, minimum width= 1cm, minimum height= 0.5cm] (a0) {a0};
        \draw[->] ([yshift= 0.3cm] a0.north west) -- (a0.north west);
        \node[above left= 0cm and -0.7cm of a0.north] {\scriptsize{\texttt{sendbuf}}};
        \node[draw, fill=orange, rectangle, right= 0.5cm of proc1, minimum width= 1cm, minimum height= 0.5cm] (a1) {a1};
        \draw[->] ([yshift= 0.3cm] a1.north west) -- (a1.north west);
        \node[above left= 0cm and -0.7cm of a1.north] {\scriptsize{\texttt{sendbuf}}};
        \node[draw, fill=orange, rectangle, right= 0.5cm of procN, minimum width= 1cm, minimum height= 0.5cm] (aN) {aN};
        \draw[->] ([yshift= 0.3cm] aN.north west) -- (aN.north west);
        \node[above left= 0cm and -0.7cm of aN.north] {\scriptsize{\texttt{sendbuf}}};

        \node[draw, fill=orange, rectangle, right= 3cm of a0, minimum width= 1cm, minimum height= 0.5cm] (a00) {a0};
        \draw[->] ([yshift= 0.3cm] a00.north west) -- (a00.north west);
        \node[above left= 0cm and -0.7cm of a00.north] {\scriptsize{\texttt{recvbuf}}};
        \node[draw, fill=orange, rectangle, right= 0cm of a00, minimum width= 1cm, minimum height= 0.5cm] (a10) {a1};
        \node[right= 0cm of a10, minimum width= 0.2cm, minimum height= 0.5cm, align=center] (etc0) {...};
        \node[draw, fill=orange, rectangle, right= 0cm of etc0, minimum width= 1cm, minimum height= 0.5cm] (aN0) {aN};

        \node[draw, fill=orange, rectangle, right= 3cm of a1, minimum width= 1cm, minimum height= 0.5cm] (a01) {a0};
        \draw[->] ([yshift= 0.3cm] a01.north west) -- (a01.north west);
        \node[above left= 0cm and -0.7cm of a01.north] {\scriptsize{\texttt{recvbuf}}};
        \node[draw, fill=orange, rectangle, right= 0cm of a01, minimum width= 1cm, minimum height= 0.5cm] (a11) {a1};
        \node[right= 0cm of a11, minimum width= 0.2cm, minimum height= 0.5cm, align=center] (etc1) {...};
        \node[draw, fill=orange, rectangle, right= 0cm of etc1, minimum width= 1cm, minimum height= 0.5cm] (aN1) {aN};

        \node[draw, fill=orange, rectangle, right= 3cm of aN, minimum width= 1cm, minimum height= 0.5cm] (a0N) {a0};
        \draw[->] ([yshift= 0.3cm] a0N.north west) -- (a0N.north west);
        \node[above left= 0cm and -0.7cm of a0N.north] {\scriptsize{\texttt{recvbuf}}};
        \node[draw, fill=orange, rectangle, right= 0cm of a0N, minimum width= 1cm, minimum height= 0.5cm] (a1N) {a1};
        \node[right= 0cm of a1N, minimum width= 0.2cm, minimum height= 0.5cm, align=center] (etcN) {...};
        \node[draw, fill=orange, rectangle, right= 0cm of etcN, minimum width= 1cm, minimum height= 0.5cm] (aNN) {aN};

        \path[] (a1.south west) -- (a01.south east) node[draw, pos=0.5, single arrow, minimum height=2cm] (allgatherarrow) {};
        \node[above left= 0cm and -0.6cm of allgatherarrow.north] {Allgather};
    \end{tikzpicture}
\end{figure}

\end{frame}

\section{Вычислительные операции}

\begin{frame}[fragile]{MPI_Reduce}

\begin{lstlisting}
int MPI_Reduce(void* sendbuf, void* recvbuf, int count, MPI_Datatype datatype, MPI_Op op, int root, MPI_Comm comm);
\end{lstlisting}

\end{frame}

\begin{frame}{Предопределенные операции в функциях редукции MPI}

\begin{table}[htp]
    \begin{center}
    \begin{tabular}{|l|l|l|}
        \hline
        Название    &   Операция        &   Поддерживаемые типы     \\
        \hline
        MPI_MAX     &   Максимум        &   integer, floating point \\
        MPI_MIN     &   Минимум         &                           \\
        \hline
        MPI_SUM     &   Сумма           &   integer, floating point \\
        MPI_PROD    &   Произведение    &   complex                 \\
        \hline
        MPI_LAND    &   И               &                           \\
        MPI_LOR     &   ИЛИ             &   integer, logical        \\
        MPI_LXOR    &   Исключающее ИЛИ &                           \\  
        \hline
    \end{tabular}
    \end{center}
\end{table}

\end{frame}

\begin{frame}{MPI_Reduce}

\begin{figure}
    \centering
    \begin{tikzpicture}
        \node[] (proc) {процесс};
        \node[below= 1cm of proc] (proc0) {0};
        \node[below= 1cm of proc0] (proc1) {1};
        \node[below= 0.6cm of proc1] (etcproc) {...};
        \node[below= 0.6cm of etcproc] (procN) {N};
        \node[right= 5cm of proc] (root) {процесс root};

        \node[draw, fill=orange, rectangle, right= 0.1cm of proc0, minimum width= 0.7cm, minimum height= 0.65cm] (a0) {a0};
        \node[draw, fill=orange, rectangle, right= 0cm of a0, minimum width= 0.7cm, minimum height= 0.65cm] (b0) {b0};
        \node[draw, fill=orange, rectangle, right= 0cm of b0, minimum width= 0.7cm, minimum height= 0.65cm] (c0) {c0};
        \draw[->] ([yshift= 0.3cm] a0.north west) -- (a0.north west);
        \node[above left= 0cm and -0.9cm of a0.north] {\scriptsize{\texttt{sendbuf}}};

        \node[draw, fill=orange, rectangle, right= 0.1cm of proc1, minimum width= 0.7cm, minimum height= 0.65cm] (a1) {a1};
        \node[draw, fill=orange, rectangle, right= 0cm of a1, minimum width= 0.7cm, minimum height= 0.65cm] (b1) {b1};
        \node[draw, fill=orange, rectangle, right= 0cm of b1, minimum width= 0.7cm, minimum height= 0.65cm] (c1) {c1};
        \draw[->] ([yshift= 0.3cm] a1.north west) -- (a1.north west);
        \node[above left= 0cm and -0.9cm of a1.north] {\scriptsize{\texttt{sendbuf}}};

        \node[draw, fill=orange, rectangle, right= 0.1cm of procN, minimum width= 0.7cm, minimum height= 0.65cm] (aN) {aN};
        \node[draw, fill=orange, rectangle, right= 0cm of aN, minimum width= 0.7cm, minimum height= 0.65cm] (bN) {bN};
        \node[draw, fill=orange, rectangle, right= 0cm of bN, minimum width= 0.7cm, minimum height= 0.65cm] (cN) {cN};
        \draw[->] ([yshift= 0.3cm] aN.north west) -- (aN.north west);
        \node[above left= 0cm and -0.9cm of aN.north] {\scriptsize{\texttt{sendbuf}}};

        \node[draw, fill=orange, rectangle, right= 2.3cm of c1.south, minimum width= 2.5cm, minimum height= 0.65cm] (opa) {op(a0,..., aN)};
        \draw[->] ([yshift= 0.3cm] opa.north west) -- (opa.north west);
        \node[above left= 0cm and 0cm of opa.north] {\scriptsize{\texttt{recvbuf}}};
        \node[draw, rectangle, right= 0cm of opa, minimum width= 0.2cm, minimum height= 0.65cm, align=center] (etc) {...};
        \node[draw, fill=orange, rectangle, right= 0cm of etc, minimum width= 2cm, minimum height= 0.65cm] (opc) {op(c0,..., cN)};

        \path[] (c1.south west) -- (opa.east) node[draw, pos=0.32, single arrow, minimum height=1.6cm] (reducearrow) {};
        \node[above left= 0cm and -0.6cm of reducearrow.north] {\texttt{Reduce}};
    \end{tikzpicture}
\end{figure}

\end{frame}

\section*{Конец}
% The final "thank you" frame 

\begin{frame}{Задания}

Реализовать параллельную сортировку массива.

Имя файла задается через аргументы командной строки.

Построить графики зависимости времени работы и ускорения от количества процессов.

\end{frame}

\begin{frame}{На следующей лекции}
\end{frame}

\begin{frame}

{\huge{Спасибо за внимание!}\par}

\vfill

\tiny{\textit{Замечание}: все торговые марки и логотипы, использованные в данном материале, являются собственностью их владельцев. Представленная здесь точка зрения отражает личное мнение автора, не выступающего от лица какой-либо организации.}

\end{frame}

\end{document}
